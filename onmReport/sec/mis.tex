\section{Management Information System}
Management information system is categorized as the use of computing and communications technology for
the safekeeping of records of the employees, their performances and company related day-to-day information.
In context of a large company, it becomes very difficult to manage and keep records of each employee, the
information about their attendance and the work completion of each employee. So, there became a need of
the automation of these clerical tasks with an aid of computer. Hence, MIS helps the company to assess the
employee as well as make any kinds of decisions based on the previous records. A good MIS gives managers
information on past and present activities and make some projections about future activities based on the past
experience. Similarly, MIS helps managers perform basic functions of the management - planning, organizing,
directing and controlling.\\
In the context of Verscend Technologies, the MIS was engaged with the following tasks in the
organization:
\begin{enumerate}
\item Digital Attendance of Each Employee: \\There was a digital attendance system installed at the entrance of
the company. It would give a card to every people who would enter the building and would directly make
an entry on the digital attendance sheet of the system.
\item Performance Evaluation of Each Employee:\\ There was a software installed on the MIS which would
assess the work performed by each employee of the organization. It applied complicated algorithms and
different statistical measures in order to evaluate the work performed by the individuals and then rated each
employee on the scale of 0-5. Managers would rely on this software for the progress and the job evaluation
of the employees under them.
\item Automated Information Provision:\\ There was a system which would check the mails automatically and
would redirect to the relevant personnel in the organization. Similarly, the system would convey any
message to any employee automatically without any human maneuver.
\end{enumerate}
\cleardoublepage
