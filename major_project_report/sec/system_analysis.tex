\newpage
\section{SYSTEM ANALYSIS}
\subsection{REQUIREMENT SPECIFICATION}
A software requirement specification is a description of a software system to be develped. It lays out functional and non-functional requirements. It describes what the software product is expected to do and what not to do. It enlists enough and necessary requirements that are required for the project development. It mainly aids to describe the scope of the work and provide a software designers a form of reference.
\subsubsection{FUNCTIONAL REQUIREMENT}
The functional requirement specification of the project are mainly categorized as user requirements, security requirements, and device requirement each of which are explained in detail below:
\begin{itemize}
\item User Requirement: User should have account on Facebook and usermust have at least one post needed to analyze the personality.
\item Security Requirement: The user can't have access to the Facebook API. User must provide their own login credentials.
\item Device Requirement: System must be initiated on web-browser.

\end{itemize}
\subsubsection{NON-FUNCTIONAL REQUIREMENT}
The non-functional requirement of the system can be summarized as follows:
\begin{itemize}
\item Performance: The system shall have a quick, accurate and reliable reuslts.
\item Capacity and Scalability: The system shall be able to store personality computed by the system into the database.
\item Availability: The system shall be avilable to user anytime whenever there is an internet connection.
\item Recovery: In case of malfunctioning or unavailability of server, the system should be able to recover and prevent any data loss or redundancy.
\item Flexibility and Portability: System shall be accessible anytime from any locations. 
\end{itemize}
\subsection{FEASIBILITY ASSESSMENT}
\subsubsection{OPERATIONAL FEASIBILITY}
\subsubsection{TECHNICAL FEASIBILITY}
\subsubsection{ECONOMIC FEASIBILITY}

