\documentclass[a4paper, 12pt, onepage]{article}
\usepackage{graphicx}
\usepackage{lipsum}
\usepackage{mathptmx}
\usepackage{hyperref}

\usepackage{fancyhdr}
\pagestyle{fancy}
\fancyhead{}
\fancyfoot{}
\fancyhead[R]{\thepage\ \hspace{1pt} }

\renewcommand{\headrulewidth}{0pt}
\renewcommand{\footrulewidth}{0pt}

\begin{document}
\pagenumbering{roman}
\addcontentsline{toc}{section}{Title Page}
\begin{titlepage}
  \centering
  \includegraphics[width=0.2\textwidth]{tu_logo.png}\par
  {\large\scshape Tribhuvan University\par}
  \vspace{0.3cm}
  {\large Pulchowk, Lalitpur \par}
  \vspace{3cm}
  	{\large\scshape A\par}
	{\large\scshape Major Project\par}
        {\large\scshape On\par}
        {\large\scshape Personality Based Music Recommendation System\par}
	\vspace{2.5cm}
        {\large\scshape Submitted by:}\\
        \vspace{0.2cm}
        {
          {\normalsize\verb+Nabin Bhattarai(070/BCT/522)+}\\
          \vspace{0.1cm}
          {\normalsize\verb+Miran Ghimire(070/BCT/521)+}\\
          \vspace{0.1cm}
          {\normalsize\verb+Brihat Ratna Bajracharya(070/BCT/513)+}\\
          \vspace{0.1cm}
          {\normalsize\verb+Abhishek Paudel(070/BCT/502)+}\\
          \vspace{0.1cm}
        }
        \vspace{1cm}
        {\large\scshape Supervised By}\\
        \vspace{0.2cm}
        {
          {\normalsize \verb+Daya Sagar Baral+\par}
          \vspace{0.1cm}
          {\normalsize \verb+Department of+\par}
          \vspace{0.1cm}
          {\normalsize\verb+Electronics and Computer Engineering+}
          \vspace{0.1cm}
        }
        
        \vspace{1cm}
        \vfill

        % Bottom of the page
	{\normalsize \today\par}
      \end{titlepage}

      \setcounter{page}{2}
      \cleardoublepage
      {
        \setlength{\parskip}{0em}
        \renewcommand\contentsname{Table of Contents}
        \tableofcontents \addcontentsline{toc}{section}{Table of Contents}
      }


      \cleardoublepage
      \pagenumbering{arabic}
      \section{Introduction}
	``\textbf{Personality Based Music Recommendation}'' is the system which uses social media profile of a person to recommend the appropriate music to that user. In this contemporary era of digital technologies, social media has become one of the prominent means for information sharing and communication. Likewise music has been one of the prominent market of entertainment. People listen to music everyday. The fact that music can blend with any emotion has made it's way to different sorts of people with different sorts of personality.\\
	Hence we have come up with the system to recommend the music to a different people based on their social media profiles.Previous work has shown that the information in users social media profiles is reflective of their actual personalities, not an idealized version of themselves, which makes social media platform for studying a people personality.\\
	Several well studied personality models have been proposed, among which the Big Five model is established as the most popular one,which suggests that the regularity in someone's behavior over time and situations uniquely identifies his/her personality type along five dimesions: Openness to experience, Neuroticsm, Extroversion, Aggreableness and Conscientiousness.

      \cleardoublepage
      \section{Summary Of Process}
      \begin{figure}[ht!]
	      \includegraphics[width=450px]{pbrs.png}
		\caption{Block Diagram of a System}
	\end{figure}
	\subsection{Task Analysis}
	\begin{center}
		\begin{tabular}{|c|l|c|c|}
			\hline
				Activity & Description & Immediate Predecessors&Time(weeks)\\
			\hline
			A&Personality DataSet Analysis&-&1\\
			\hline
			B&Feature Extraction from Personality DataSet&A&3\\
			\hline
			C&Personality Analysis Model&B&4\\
			\hline
			D&User Social Media Information Extraction&-&1\\
			\hline
			E&User Personality Analysis&D,C&2\\
			\hline
			F&Music DataSet Analaysis&-&1 \\
			\hline
			G&Feature Extraction from Music DataSet&F&2\\
			\hline
			H&Recommendation Model&G,E&4\\
			\hline
			I&Testing and Debugging&H&2\\
			\hline

		\end{tabular}
	\end{center}
	\cleardoublepage
	\subsection{Task Evaluation}
		\subsubsection{Task Completed}
		\begin{enumerate}
		\item Personality DataSet Analysis
		\item Feature Extraction from Personality DataSet
		\item Personality Analysis Model
		\end{enumerate}

		\subsubsection{Task Remaining}
		\begin{enumerate}
		\item User Social Media Information Extraction
		\item User Personality Analysis
		\item Music DataSet Analysis
		\item Feature Extraction from Music DataSet
		\item Recommendation Model
		\item Testing and Debugging
		\end{enumerate}
      \subsection{Schedule Analysis}
      \begin{figure}[ht!]
	      \includegraphics[width=450px]{aschedule.png}
	      \caption{PERT Activity of Node Diagram of Project}
	\end{figure}
	\hfill \break
	Critical path: A-B-C-E-H-I\\
	Estimate time of completion: 16 weeks\\
	Time Elapsed: 4 weeks
	\subsection{Progress Analysis}
	Activity to be completed at this time = A-B-C (4 weeks)\\
	Activity Completed = A-B-C\\
	We are almost within a schedule.
      %\subsection{Budget Analysis}
      %\subsection{Scope Analysis}
      %\subsection{Process Analysis}
      %\subsection{Gantt Progress Chart}

      \section{Activity Analysis}
      \subsection{Task completed till date}
	\begin{center}
		\begin{tabular}{|c|l|c|c|}
			\hline
				Activity & Description & Immediate Predecessors&Time(weeks)\\
			\hline
			A&Personality DataSet Analysis&-&1\\
			\hline
			B&Feature Extraction from Personality DataSet&A&3\\
			\hline
			C&Personality Analysis Model&B&4\\
			\hline
		\end{tabular}
	\end{center}
	\begin{enumerate}
		\item \textbf{Personality DataSet Analysis:}\\
		In order to predict a personality of social media users, myPersonality dataset was used. Dataset was obtained from: \url{http://mypersonality.org/wiki/doku.php?id=download_databases} \\
		It consisted of:
		\begin{enumerate}
		\item AUTHID : \\ 
			User Id, it was represented with a unique random number in order to protect user's identity.
      \begin{figure}[h!]
	      \centering
	      \includegraphics[width=100px]{userid.png}
		\caption{AuthId in dataset}
	\end{figure}
		\item STATUS : \\ 
			It consisted of status posted by the user's in certain time period frame. It consisted of total of 9918 status post of more than 250 users.
      \begin{figure}[h!]
	      \includegraphics[width=450px]{status.png}
		\caption{status in dataset}
	\end{figure}
		\item PERSONALITY CLASSIFICATION : \\
			It classified personality as Big Five Personality traits which consisted of:\\
		\begin{enumerate}
		\item Openness to Experience:
			curious, intelligent, imaginative. High scorers
			tend to be artistic and sophisticated in taste and appreciate diverse views,
			ideas and experiences.
		\item Conscientiousness:
			responsible, organized, preservering. Conscientious
			individuals are extremely realiable and tend to be high achievers, hard work-
			ers and planners.
		\item Extroverion:
			outgoing, amicable, assertive. Friendly and energetic, ex-
			troverts draw inspiration from social situations.
		\item Agreeableness:
		cooperative, helpful,nurturing. People who score high in
		agreeableness are peace-keepers who are generally optimistic and trusting of
		others.
		\item Neuroticism:
			anxious, insecure, sensitive. Neurotics are moody, tense and
			easily tipped into experiencing negative emotions.
		\end{enumerate}
      \begin{figure}[h!]
	      \centering
	      \includegraphics[width=100px]{personality.png}
		\caption{personality classification in dataset}
	\end{figure}
		\end{enumerate}

	\clearpage
	\item Feature Extraction from Personality DataSet:\\
	\begin{enumerate}
	\item Bag of Words:\\
		It is a one of the technique of ``Data Preprocessing'' in NLP in which each word in a document is treated as the feature. It consist of two models:
	\begin{itemize}
		\item Continuous-Gram Model:\\
			It is the model in which every single words that appears in the document is treated as the feature in the sequence in which theyappear. In this model both the presence of the word and appearance sequence both matters.
		\item Skip-Gram Model:\\
			It is the model in which every single words that appears in the document is treated as the feature. However the sequence in which they appear doesn't matter.
	\end{itemize}
	The bag of words model is the most commonly used methods in sentence classification where the frequency of occurence of each word is used as a feture for training a classifier. After the transforming the text into the bag of words, we calculate various measures to characterize the text. The most common type of characteristics calculated from the bag of word is term-frequency i.e the number of times the term appears in the text.\\
	In case, of our project, skip-gram model has been used because the existence of the word in a status posted by the user matters not the sequence in which they appear. After the bag of words extraction, term-frequency has been calculated for the classification purpose.
	\item Stop Words:\\
		It is also one of the technique of ``Data Preprocessing'' in NLP. It is commonly used with bag of words or any kind of models like n-gram in order to elimate the words that are of no use or very less use in the classification purpose. Such as article, preposition etc. The stop word vary within the project to project and depends on specific scenario.\\
		In case of our, project stop words from nltk module has been used. It consist prepostion, article. Besides of the additional stop words also has been added as per the requirement of the project.
	\clearpage
      \begin{figure}[ht!]
	      \includegraphics[width=600px]{stopWords.png}
		\caption{stopwords used in the project}
	\end{figure}
	\end{enumerate}
	\item Personality Analysis Model
	\end{enumerate}
	\subsection{Current Deliverable}
	\begin{itemize}
	\item Personality Prediction Model
	\end{itemize}
    %  \subsection{Short term future tasks and deliverable}

    %  \cleardoublepage
    % \section{Previous problems and issues}
    %  \subsection{Action item and status} %k vanya maile bujhena hai
    %  \subsection{New or revised action items} %yo ni

      \cleardoublepage
      \section{Current problems and issues}
      \subsection{Problems}
      \subsection{Issues}
      \subsection{Possible solutions}
      \subsubsection{Recommendation}
  %    \subsubsection{Assignment of responsibility}
  %    \subsubsection{Deadline}
      \section{Task to be completed in Future}
	\begin{center}
		\begin{tabular}{|c|l|c|c|}
			\hline
				Activity & Description & Immediate Predecessors&Time(weeks)\\
			\hline
			D&User Social Media Information Extraction&-&1\\
			\hline
			E&User Personality Analysis&D,C&2\\
			\hline
			F&Music DataSet Analaysis&-&1 \\
			\hline
			G&Feature Extraction from Music DataSet&F&2\\
			\hline
			H&Recommendation Model&G,E&4\\
			\hline
			I&Testing and Debugging&H&2\\
			\hline
		\end{tabular}
	\end{center}
\end{document}

%%% Local Variables:
%%% mode: latex
%%% TeX-master: t
%%% End:
